\documentclass[a4paper]{article}

%% Language and font encodings
\usepackage[english]{babel}
\usepackage[utf8x]{inputenc}
\usepackage[T1]{fontenc}

%% Useful packages
\usepackage{amsmath}
\usepackage{amssymb}
\usepackage{amsfonts} 
\usepackage{graphicx}
\usepackage{listings}
\usepackage{color}
\usepackage[colorinlistoftodos]{todonotes}
\usepackage[colorlinks=true, allcolors=blue]{hyperref}

%% Defining commands
\newcommand{\R}{\mathbb{R}}
\newcommand{\Z}{\mathbb{Z}}
\newcommand{\Q}{\mathbb{Q}}
\newcommand\tab[1][1cm]{\hspace*{#1}}
\newcommand*{\perm}[2]{{}^{#1}\!P_{#2}}%
\newcommand*{\comb}[2]{{}^{#1}C_{#2}}%

%% Sets page size and margins
\usepackage[a4paper,top=3cm,bottom=2cm,left=3cm,right=3cm,marginparwidth=1.75cm]{geometry}

%% Setting up code blocks
\lstset{frame=tb,
	aboveskip=3mm,
	belowskip=3mm,
	showstringspaces=false,
	columns=flexible,
	basicstyle={\small\ttfamily},
	numbers=none,
	breaklines=true,
	breakatwhitespace=true,
	tabsize=3
}




\title{%
	CS1231 Part 4 - Number Theory  \\
	\large Based on lectures by Terence Sim and Aaron Tan
	\\ Notes taken by Andrew Tan
	\\ AY18/19 Semester 1
	\\ }

\author{}
\date{\vspace{-5ex}}

\begin{document}
\maketitle

\begin{center}\begin{minipage}[c]{0.9\textwidth}\centering\footnotesize These notes are not endorsed by the lecturers, and I have modified them (often significantly) after lectures. They are nowhere near accurate representations of what was actually lectured, and in particular, all errors are almost surely mine.\end{minipage}\end{center}

\section{Mathematical Induction}
The Principle of Mathematical Induction is an inference rule concerning a predicate $P(n)$:
\begin{multicols}{3}
\flushleft Base case:  \\
Inductive step: \\
Conclusion:\\
\columnbreak
$P(0)$\\
$\forall k \in \N, P(k) \rightarrow P(k+1)$\\
$\bullet \forall n \in \N, P(n)$
\columnbreak
\end{multicols}
The steps for using mathematical induction are outlined as such:
\begin{enumerate}
	\item Identify the predicate $P(n)$. The predicate is a statement that evaluates to true or false. Usually, $n\in\N$, and in any case, we need to qualify the domain of $n$ by saying "For all $n\in\N"$ or the respective domain.
	\item Prove the \textbf{Base case}, P(0). Note that there can be more than one base case, and it need not start at P(0).
	\item Prove the \textbf{Inductive step}, which is an implication statement involving universal quantification. The usual rules for proving such statements apply here, and should have the following steps:\\ \\
	For any $k\in\N$:
	\begin{enumerate}
		\item[3.1] Assume $P(k)$ is true [Denoted as the \textit{Inductive hypothesis}]
		\item[3.2] Consider P(k+1), and break it down into a smaller problem of size k.
		\item[3.3] Apply the inductive hypothesis on the size-k problem.
		\item[3.4] Proceed to show that P(k+1) is true.
	\end{enumerate}
	\item Write the \textbf{Conclusion} (Given that the base case $P(0)$ is true, it follows that $P(1)$ is true and so on.)
\end{enumerate}
\subsection{Strong induction}
The only difference between Strong Induction and Regular Induction lies only in the Inductive hypothesis.\\
In Strong Induction, we assume $P(k), P(k-1), P(k-2),\dots,P(a)$ are \textit{all} true.\\
Essentially, we're making a stronger assumption about the values of $n$ which make $P(n)$ true, from this stronger assumption, we proceed as before to show that $P(k+1)$ is true.

\section{Prime numbers}
An integer $n$ is \textbf{prime} if, and only if, $n>1$ and for all positive integers $r$ and $s$, if $n=rs$, then either $r$ or $s$ equals $n$.\\
An integer $n$ is \textbf{composite} if, and only if, $n>1$ and $n=rs$ for some integers $r$ and $s$ with $1<r<n$ and $1<s<n$.\\ \\ \\
Symbolically,\\
\begin{tabular}%
{>{\raggedleft\arraybackslash}p{3.5cm}%
	>{\raggedright\arraybackslash}p{10cm}%
}
$n$ is prime $\iff$ & $\forall$ positive integers $r$ and $s$, if $n=rs$ then either $r=1$ and $s=n$ or $r=n$ and $s=1$\\
$n$ is composite$\iff$ & $\exists$ positive integers $r$ and $s$ such that $n=rs$ and $1<r<n$ and $1<s<n$
\end{tabular}\\ \\
Clearly, every integer $n>1$ is either prime or composite.

\subsection{The Fundamental Theorem of Arithmetic}
The Fundamental Theorem of Arithmetic states that every positive integer greater than 1 can be \textit{uniquely} factorized into a product of prime numbers.\\ \\
More formally, given any integer $n>1$, there exists a positive integer $k$, distinct prime numbers $p_1,p_2,\dots,p_k$ and positive integers $e_1,e_2,\dots,e_k$ such that
\begin{center} $n=p^{e_1}_{1}p^{e_2}_{2}p^{e_3}_{3}\dots p^{e_k}_{k}$,
\end{center}
and any other expression for $n$ as a product of primes is identical to this except, perhaps, for the order in which the factors are written.

\subsection{Primality test}
There are multiple tests to see if an integer $n$ is prime.\\ \\
The most straightforward method is Trial Division, by testing if $n$ is divisible by all integers $k$ between $2$ and $\sqrt{n}$.\\ \\
The Sieve of Eratosthenes is a list of primes that is generated simply by starting with a list $C$ of all integers greater than 1 and $p=2$, and crossing out all multiples of $p$, and repeating with the next uncrossed number in $C$.\\ \\ 
The Miller-Rabin primality test is another primality test which determines whether a given number is prime. It relies on a set of equalities that hold true for prime values. However, it is probabalistic, and composites may be passed off as a prime.

\subsection{Open questions}
There are several open questions concerning prime numbers, and listed below are a few of interest:\\ \\
\textbf{Goldbach's Conjecture}: Every even integer greater than 2 can be written as a sum of two primes.\\
\textbf{Twin Primes Conjecture}: There are infinitely many primes $p$ such that $p+2$ is also a prime.


\appendix
\section{Prime properties}
\subsection{Theorems}
\begin{itemize}
	\item[] Theorem 4.2.3: If $p$ is a prime and $x_1, x_2,\dots, x_n$ are any integers such that: $p\mid x_1x_2\dots x_n$, then $p\mid x_i$ for some $x_i$ $(1\le i \le n)$.
	\item[] Theorem 4.3.5 (Epp): Fundamental Theorem of Arithmetic: Given any integer $n>1$, there exists a positive integer $k$, distinct prime numbers $p_1,p_2,\dots,p_k$ and positive integers $e_1,e_2,\dots,e_k$ such that $n=p^{e_1}_{1}p^{e_2}_{2}p^{e_3}_{3}\dots p^{e_k}_{k}$, and any other expression for $n$ as a product of primes is identical to this except, perhaps, for the order in which the factors are written.
	\item[] Theorem 4.7.3 (Epp): The set of primes is infinite.
	\item[] Prime Number Theorem: The number of primes less than or equal to an integer $x$ is approximately $x /\log(x)$.
\end{itemize}
\subsection{Propositions}
\begin{itemize}
	\item[] Proposition 4.2.2: For any two primes $p$ and $p'$, if $p\mid p'$ then $p=p'$.
	\item[] Proposition 4.7.3 (Epp): For any $a\in\Z$ and any prime $p$, if $p\mid a$ then $p\nmid a$ then $p\mid (a+1)$.
\end{itemize}


\end{document}