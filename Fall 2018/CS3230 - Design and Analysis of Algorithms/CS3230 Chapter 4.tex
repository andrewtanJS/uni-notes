\documentclass[a4paper]{article}

%% Language and font encodings
\usepackage[english]{babel}
\usepackage[utf8x]{inputenc}
\usepackage[T1]{fontenc}

%% Useful packages
\usepackage{amsmath}
\usepackage{amssymb}
\usepackage{amsfonts} 
\usepackage{graphicx}
\usepackage{listings}
\usepackage{varwidth}
\usepackage{multicol}
\usepackage{array}
\usepackage{color}
\usepackage[colorinlistoftodos]{todonotes}
\usepackage[colorlinks=true, allcolors=blue]{hyperref}

%% Defining commands
\newcommand{\R}{\mathbb{R}}
\newcommand{\Z}{\mathbb{Z}}
\newcommand{\Q}{\mathbb{Q}}
\newcommand{\N}{\mathbb{N}}
\newcommand\tab[1][1cm]{\hspace*{#1}}
\newcommand*{\perm}[2]{{}^{#1}\!P_{#2}}%
\newcommand*{\comb}[2]{{}^{#1}C_{#2}}%

%% Sets page size and margins
\usepackage[a4paper,top=3cm,bottom=2cm,left=3cm,right=3cm,marginparwidth=1.75cm]{geometry}

%% Setting up code blocks
\lstset{frame=tb,
	aboveskip=3mm,
	belowskip=3mm,
	showstringspaces=false,
	columns=flexible,
	basicstyle={\small\ttfamily},
	numbers=none,
	breaklines=true,
	breakatwhitespace=true,
	tabsize=3
}




\title{%
	CS3230 Chapter 4 - Data Structures \\
	\large Based on lectures by Chang Ee-Chien
	\\ Notes taken by Andrew Tan
	\\ AY18/19 Semester 1
	\\ }

\author{}
\date{\vspace{-5ex}}

\begin{document}
\maketitle

\begin{center}\begin{minipage}[c]{0.9\textwidth}\centering\footnotesize These notes are not endorsed by the lecturers, and I have modified them (often significantly) after lectures. They are nowhere near accurate representations of what was actually lectured, and in particular, all errors are almost surely mine.\end{minipage}\end{center}

\section{Data structures}
A data structure is simply an organized method of storing and retrieving data. \\ \\Here we will look at 3 different data structures: priority queues, binary search trees, and hash tables.
\subsection{Priority Queue - Heap}
A \textbf{priority queue} is a data-structure that maintains a collection of elements and supports the following three operations:
\begin{itemize}
	\item Insert: Insert an element
	\item ExtractMin: Remove the smallest element
	\item ReduceKey: Reduce the value of one element
\end{itemize}
A \textbf{heap} is a tree where any parent is smaller or equal to its children. The heap will allow us to implement a priority queue with its operations performing efficiently.\\ \\
The heap itself is implemented as a binary tree, which is itself implemented with an array, where
\begin{itemize}
	\item The parent of $A[i]$ is $A[\lfloor i/2 \rfloor]$
	\item The left child of $A[i]$ is $A[2i]$
	\item The right child of $A[i]$  is $A[2i + 1]$
\end{itemize}
The element without a parent is the root. Thus, $A[1]$ is the root. Furthermore, an element without children is a leaf.\\ \\
Thus, let $A$ be an array of $n$ elements representing a binary tree. $A$ is a heap if
\begin{center}
	$A[\lfloor i/2 \rfloor] \le A[i]$ for all $2 \le i \le n$
\end{center}
The two basic operations on the heap are \textit{SiftUp} and \textit{SiftDown}. SiftUp and SiftDown allows the array to be modified if a value in the heap is changed to a smaller or larger value respectively, so that it is again a heap.\\ \\
Iterative version of SiftUp, where A is the array, i is the value to SiftUp.
\begin{lstlisting}[escapeinside={(*}{*)}]
SiftUp((*$A$*), (*$i$*)):
	if (*$i==1$*) then return
	(*$j \leftarrow \lfloor i/2 \rfloor;$*) # where j is the parent
	if (*$A[j] <= A[i]$*) then return
	swap ((*$A[i]$*), (*$A[j]$*))
	SiftUp((*$A$*), (*$j$*))
\end{lstlisting}
Iterative version of SiftDown, where A is the array, i is the value to SiftDown, n is the size of the heap.
\begin{lstlisting}[escapeinside={(*}{*)}]
SiftDown((*$A$*), (*$i$*), (*$n$*)):
	(*$j \leftarrow 2i$*)
	while (*$j <= n$*):
		if (*$j<n$*) and (*$A[j+1]<A[j]$*):
			(*$j\leftarrow j+1$*)
		if (*$A[i]$*) <= (*$A[j]$*) then return;
		swap ((*$A[i]$*), (*$A[j]$*))
		(*$i \leftarrow j$*)
		(*$j \leftarrow 2i$*)
\end{lstlisting}
To implement a priority queue, we can use SiftUp and SiftDown:
\begin{lstlisting}[escapeinside={(*}{*)}]
insert((*$H$*), (*$key$*)):
	(*$H[heap.size + 1] \leftarrow key$*)
	SiftUp((*$H$*), (*$heap.size + 1$*))
	(*$heap.size$*)++
\end{lstlisting}
\begin{lstlisting}[escapeinside={(*}{*)}]
extractMin((*$H$*)):
	(*$v \leftarrow H[1]$*)
	(*$H[1] \leftarrow H[heap.size]$*)
	(*$heap.size \leftarrow heap.size-1$*)
	SiftDown((*$H$*), (*$heap.size$*), (*$1$*))
	return (*$v$*)
\end{lstlisting}
The complexity of SiftDown and SiftUp are in $O(\log(n))$, where $n$ is the total number of elements in the array. Therefore, the complexity of insert and extractMin are in $O(\log(n))$
\subsubsection{Heap sort}
With the priority queue, we can now implement HeapSort:
\begin{lstlisting}[escapeinside={(*}{*)}]
HeapSort((*$A$*), (*$n$*)):
	Let (*$H$*) be the empty heap
	for (*$i \leftarrow 1$*) to (*$n$*):
		insert((*$H$*), (*$A[i]$*))
	for (*$i \leftarrow 1$*) to (*$n$*):
		(*$A[i] \leftarrow$*) extractMin((*$H$*))
\end{lstlisting}
The running time of HeapSort is:\\
$T(1) + T(2) + \dots + T(n-1) + T(n)$\\
$ + K(n) + K(n-1) + \dots + K(2) + K(1)$\\
$<nT(n) + nK(n)$\\
where $T(n)$ and $K(n)$ are the running times of insert and extractMin respectively.\\ \\
Thus, the running time of HeapSort is in $O(n\log(n))$
\subsubsection{Make heap}
We can sort $n$ elements in decreasing order without creating a temporary array by making a heap from the input array and then swapping accordingly:
\begin{lstlisting}[escapeinside={(*}{*)}]
MakeHeapTopDown((*$A$*), (*$n$*)):
	for (*$i \leftarrow 2$*) to (*$n$*):
		SiftUp((*$A$*), (*$i$*))
\end{lstlisting}
The idea is that we start off with a heap, and swap the top of the heap (i.e. the smallest element) with the last element of the array, and sift down accordingly. This splits the array into two sections: the heap and the sorted array.
\begin{lstlisting}[escapeinside={(*}{*)}]
HeapSortDecreasing((*$A$*), (*$n$*)):
	MakeHeap((*$A$*), (*$n$*)
	for (*$i \leftarrow 1$*) to (*$n$*):
		swap((*$A[1]$*), (*$A[n - 1 + i]$*))
		SiftDown((*$A$*), (*$n-i$*), (*$i$*))
\end{lstlisting}
We can also MakeHeap recursively, by heapifying the left and right subtree recursively, and then perform a SiftDown.\\\\
Let $T(n)$ be the running time of MakeHeap, where $n$ is the number of elements. For simplicity, assume $n=2^k + 1$ for some $k$.
\begin{center}
	$T(n) = 2T(\frac{n-1}{2}) + f(n)$ where $f(n) \in \Theta(\log(n))$\\
\end{center}
Approximating the above by\\
\begin{center}
	$T(n) = 2T(\frac{n}{2}) + f(n)$ where $f(n) \in \Theta(\log(n))$,\\
\end{center}
we can use the Master Theorem to obtain $T(n) \in \Theta(n)$\\ \\
MakeHeap also works from the bottom up, as such:
\begin{lstlisting}[escapeinside={(*}{*)}]
MakeHeapBottomUp((*$A$*), (*$n$*)):
	for (*$i \leftarrow \lfloor n/2 \rfloor$*) to (*$1$*):
	SiftDown((*$A$*), (*$i$*), (*$n$*))
\end{lstlisting}
The number of comparisons here is less than
\begin{center}
	$c\sum\limits_{j=1}^{k}j2^{k-j}$
\end{center}
where $c$ is some constant, $k=\log(n)$, and $j$ represents the number of comparisons in a sub-tree at 'level $j$', and $2^{k-j}$ represents the number of sub-trees at that level.\\
Simplifying the above, we obtain a running time in $\Theta(n)$.

\section{Binary search tree}
A binary tree is a \textbf{binary search tree} if the values contained in every vertex is larger than or equal to the values contained in its left-subtree, and less than or equal to the values contained in its right subtree.\\
\subsection{Tree search}
Tree search is as simple as recursing into the appropriate subtree until we find the key.
\subsection{Ordered traversal}
There are multiple ways to systematically traverse a binary tree:\\ \\
Given a binary tree root $T$, we recursively traverse the tree and print accordingly:
\begin{multicols}{3}
\textbf{Pre-order}:
\begin{itemize}
	\itemsep0em
	\item print current node
	\item traverse left subtree
	\item traverse right subtree
\end{itemize}
\textbf{In-order}:
\begin{itemize}
	\itemsep0em
	\item traverse left subtree
	\item print current node
	\item traverse right subtree
\end{itemize}
\textbf{Post-order}:
\begin{itemize}
	\itemsep0em
	\item traverse left subtree
	\item traverse right subtree
	\item print current node
\end{itemize}
\end{multicols}
\subsection{Deletion in a BST}
When deleting a node $z$ from a BST $T$, the updated tree must still be a BST.\\
If $z$ has no children, then the problem is trivial.\\
If $z$ has one child, then we simply have the parent of $z$ point to the child of $z$.\\
If $z$ has two children, then we must first find the successor $v$ of $z$, and replace $z$ with $v$.
\subsection{Balanced BST - red-black tree}
A binary search tree supports deletion, insertion, and searching in $O(h)$ time, where $h$ is the height of the tree.\\
The largest possible $h$ is in $\Theta(n)$ where $n$ is the total number of elements.\\
If we keep the binary tree balanced (also called a \textbf{red-black tree}), that is, for any node, the difference in height between the left and right subtree is at most one, then $h$ is in $\Theta(\log(n))$.

\section{Hash table}
A hash table is a data structure that supports insertion, deletion, and searching, where, under reasonable assumptions, the expected time is in $\Theta(1)$.\\ \\
Let $U$ be the set of all possible keys, instead of creating an array of size |$U$|, we create a hash table $H$ of a smaller size, say $M$.\\
A hash function $h$, is a function from $U$ to \{$0,\dots,M-1$\} (which we assume can be computed in $O(1)$, and a key $k$ is stored in a slot $h(k)$.
\subsection{Collision}
If we wish to hash a key $k$ in a slot $h(k)$ that is already occupied, we say that a \textbf{collision} occurred. \textbf{Collision resolutions} are methods to resolve such collisions.
\subsubsection{Resolution 1 - Separate chaining}
Instead of having each index of the hash table store a key, we have them store a linked list of keys, and append them accordingly.\\\\
Suppose a hash table $H$ has $M$ slots, and it stores $n$ elements. We define the \textbf{load factor} for $H$ as $(n/M)$. We assume \textbf{simple uniform hashing}, that is, any given key is equally likely to hash into any of the $M$ slots, independent of where any other element has hashed to.\\\\
Thus, the expected number of keys hashed to a particular slot is $a$, where $a$ is the load factor. Therefore, the expected total time required to search for a key $k$ is $O(1+a)$.

\subsubsection{Resolution 2 - Open addressing}
When collision occurs, we store the key in another available slot in the hash table. There are several methods to choose the available slot:
\begin{itemize}
	\item[A] \textbf{Linear Probing} When collision occurs, store the key in the next available slots greater than $h(k)$. That is, search through $h(k)+1, h(k)+2, \dots$.
	\item[B] \textbf{Quadratic Probing} When collision occurs, store the key in the next available slot according to this sequence: $h(k)+1^2, h(k)+2^2, \dots$.
	\item[C] \textbf{Double Hashing}
	Let $h(k, i)$ be a hash function with two variables. When collision occurs, store the key in the next available slot according to this sequence: $h(k, 0), h(k, 1), h(k, 2), \dots$.
\end{itemize}
Given a key, the expected number rof probes required to know that key is not in the hash table is at most $1/(1-a) \approx 1 + a + a^2 + a^3 + \dots$\\
Likewise, given a key in the hash table, the expected number of probes required to find the location of the key is at most $(1/a) \ln(1-a)^{-1}$

\appendix
\section{Graph and tree definitions}
\begin{itemize}
	\item A \textbf{directed graph} is a pair $G=\langle V,E\rangle$ where $V$ is a set of \textbf{vertices} and $E \subseteq V \times V$ is a set of \textbf{edges}. If $(v, w) \in E$ then we say there is an edge from vertex $v$ to vertex $w$.
	\item A \textbf{path} from $v$ to $u$ is a sequence of edges leaving from $v$ to $u$.
	\item A \textbf{rooted tree} is a directed graph where there exists a vertex $r$ such that every other vertex can be reached from $r$ by a unique path. Hence $r$ is denoted the \textbf{root}.
	\item In a tree, if $(v, w)$ is an edge, we say that $v$ is the \textbf{parent} of $w$, and $w$ is the \textbf{child} of $v$.
	\item If there is a directed path from $v$ to $w$, we say that $w$ is the \textbf{descendant} of $v$.
	\item If a vertex has no children, it is a \textbf{leaf}. Otherwise, it is an \textbf{internal node}.
	\item If every vertex in the rooted tree has at most $n$ children, then we say that this tree is a \textbf{n-ary tree}.
	\item The \textbf{height} of the tree is the number of edges along the longest path from the root to the leaves.
	\item A \textbf{binary tree} is a 2-ary tree. The two children of a binary tree are the \textbf{left-hand} child and \textbf{right-hand} child.
	\item The \textbf{predecessor} of a node $p$ is the node occurring immediately before $p$ in the in-order traversal of the tree.
	\item The \textbf{successor} of a node $p$ is the node occuring immediately after $p$ in the in-order traversal of the tree.
\end{itemize}

\end{document}