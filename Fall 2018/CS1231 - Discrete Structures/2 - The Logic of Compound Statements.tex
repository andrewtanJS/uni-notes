\documentclass[a4paper]{article}

%% Language and font encodings
\usepackage[english]{babel}
\usepackage[utf8x]{inputenc}
\usepackage[T1]{fontenc}

%% Useful packages
\usepackage{amsmath}
\usepackage{amssymb}
\usepackage{amsfonts} 
\usepackage{graphicx}
\usepackage{listings}
\usepackage{varwidth}
\usepackage{multicol}
\usepackage{array}
\usepackage{color}
\usepackage[colorinlistoftodos]{todonotes}
\usepackage[colorlinks=true, allcolors=blue]{hyperref}

%% Defining commands
\newcommand{\R}{\mathbb{R}}
\newcommand{\Z}{\mathbb{Z}}
\newcommand{\Q}{\mathbb{Q}}
\newcommand{\N}{\mathbb{N}}
\newcommand\tab[1][1cm]{\hspace*{#1}}
\newcommand*{\perm}[2]{{}^{#1}\!P_{#2}}%
\newcommand*{\comb}[2]{{}^{#1}C_{#2}}%

%% Sets page size and margins
\usepackage[a4paper,top=3cm,bottom=2cm,left=3cm,right=3cm,marginparwidth=1.75cm]{geometry}

%% Setting up code blocks
\lstset{frame=tb,
	aboveskip=3mm,
	belowskip=3mm,
	showstringspaces=false,
	columns=flexible,
	basicstyle={\small\ttfamily},
	numbers=none,
	breaklines=true,
	breakatwhitespace=true,
	tabsize=3
}




\title{%
	CS1231 Part 2 - The Logic of Compound Statements  \\
	\large Based on lectures by Terence Sim and Aaron Tan
	\\ Notes taken by Andrew Tan
	\\ AY18/19 Semester 1
	\\ }

\author{}
\date{\vspace{-5ex}}

\begin{document}
\maketitle

\begin{center}\begin{minipage}[c]{0.9\textwidth}\centering\footnotesize These notes are not endorsed by the lecturers, and I have modified them (often significantly) after lectures. They are nowhere near accurate representations of what was actually lectured, and in particular, all errors are almost surely mine.\end{minipage}\end{center}

\section{Logical form and logical equivalence}
\subsection{Statements}
A \textbf{statement} (or \textbf{proposition}) is a sentence that is true or false, but not both.
\subsubsection{Compound statements}
\textbf{Negation} - if $p$ is a statement variable, the negation of $p$ is "not $p$" and is denoted $\sim p$.\\
\textbf{Conjunction} - if $p$ and $q$ are statement variables, the conjunction of $p$ and $q$ is "$p$ and $q$", denoted $p \land q$.\\
\textbf{Disjunction} - if $p$ and $q$ are statement variables, the disjunction of $p$ and $q$ is "$p$ or $q$", denoted $p \lor q$.\\
\subsubsection{Statement form}
A \textbf{statement form} (or \textbf{propositional form}) is an expression made up of statement variables and logical connectives that becomes a statement when actual statement are substituted for the component statement variables.
\subsection{Logical equivalence}
Two statement forms are called \textbf{logically equivalent} if, and only if, they have identical truth values for each possible substitution of statements for their statement variables.\\ \\
The logical equivalence of statement forms $P$ and $Q$ is denoted by $P \equiv Q$
\subsubsection{Non-equivalence}
To show that statement forms $P$ and $Q$ are \textit{not} logically equivalent, show that:
\begin{enumerate}
	\item At least one row in both their truth tables where their truth values differ, or
	\item Find a counter example - concrete statements for both of the two forms, one of which is true and the other of which is false. 
\end{enumerate}
\subsection{Tautologies and Contradictions}
A \textbf{tautology} is a statement form that is always true regardsless of the truth values of the individual statements substituted for its statement variables.\\
A tautological statement is a statement whose form is a tautology.\\ \\
A \textbf{contradiction} is a statement form that is always false regardless of the truth values of the individual statements substituted for its statement variables.\\
A contradictory statement is a statement whose form is a contradiction.
\section{Conditional statements}
A conditional statement takes the form of "if $p$ then $q$", or $p \rightarrow q$.\\
It is false when $p$ is true and $q$ is false; otherwise it is true.\\

A conditional statement that is true simply because the hypothesis is false is called \textit{vacuously true}
We call $p$ the hypothesis (or antecedent) of the conditional and $q$ the conclusion (or consequent).
\subsection{Implication law}
We may represent the conditional statement as an or statement:
\begin{center}
	$p \rightarrow q$ $\equiv$ $\sim p \lor q$
\end{center}
Both statement forms are logically equivalent, and this equivalence is known as the implication law.
\subsection{Contrapositive}
The contrapositive of a conditional statement of the form "if $p$ then $q$" is 
\begin{center}
	if $\sim q$ then $p$
\end{center}
Recall that a statement and its contrapositive are logically equivalent.

Also, to say "$p$ \textit{only if} $q$" means that $p$ can take place only if $q$ takes place as well. It is logically equivalent to the statement form "if not $q$ then not $p$" and likewise "if $p$ then $q$".
\subsection{Converse}
The converse of a conditional statement of the form "if $p$ then $q$" is
\begin{center}
	if $q$ then $p$
\end{center}
\subsection{Inverse}
The inverse of a conditional statement of the form "if $p$ then $q$" is
\begin{center}
	if $\sim p$ then $\sim q$
\end{center}
Note that given a conditional statement, its converse and inverse are logically equivalent, \textbf{but} the statement itself is not logically equivalent to both its converse or inverse.
\subsection{Biconditional}
Given two statement variables $p$ and $q$, the \textbf{biconditional} of $p$ and $q$ is "$p$ if, and only if, $q$" and is denoted $p \leftrightarrow q$. It is true if both $p$ and $q$ have the same truth values, and false if $p$ and $q$ have opposite truth values.\\
\\The words \textit{if and only if} are sometimes abbreviated as \textit{iff}
\subsection{Necessary and sufficient conditions}
If $r$ and $s$ are statements,\\
"$r$ is a sufficient condition for $s$" means "if $r$ then $s$"\\
"$r$ is a necessary condition for $s$" means "if not $r$ then not $s$"
\section{Valid and invalid arguments}
An argument is a sequence of statements ending in a conclusion.\\
All statements in an argument, except for the final one, are called \textbf{premises} (or \textbf{assumptions} or \textbf{hypothesis}). The final statement is called the \textbf{conclusion}.\\
The symbol $\bullet$ is read as "therefore", and is normally placed just before the conclusion.\\ \\
An argument form is called \textbf{valid} if, and only if, whenever statements are substituted that make all the premises true, the conclusion is also true.\\
When an argument is valid and its premises are true, the truth of the conclusion is said to be \textbf{inferred} or \textbf{deduced} from the truth of the premises.
\subsection{Determining validity}
To test an argument form for validity:
\begin{enumerate}
	\item Identify the premises and conclusion of the argument form.
	\item Construct a truth table showing the truth values of all the premises and the conclusion.
	\item If there is a critical row (A row of the truth table in which all the premises are true) in which the conclusion is 
	\subitem \textbf{true}, then the argument form is valid.
	\subitem \textbf{false}, then the argument form is invalid.
\end{enumerate}
\subsection{Syllogism}
A \textbf{Syllogism} is an argument form consisting of two premises and a conclusion.
\subsubsection{Modus ponens}
Modus ponens is a valid syllogism that takes the form:
\begin{itemize}
	\itemsep0em
	\item[] if $p$ then $q$
	\item[] $p$
	\item[] $\bullet q$
\end{itemize}
\subsubsection{Modus Tollens}
Likewise, modus tollens is a valid syllogism that takes the form:
\begin{itemize}
	\itemsep0em
	\item[] if $p$ then $q$
	\item[] $\sim q$
	\item[] $\bullet \sim p$
\end{itemize}
Note that modus tollens takes advantage of the contrapositive of a statement.
\subsection{Rules of inference}
A rule of inference is a form of argument that is valid.\\ \\
Along with modus ponens and modus tollens, there are other rules of inference.
\subsubsection{Generalization}
\begin{varwidth}[t]{.5\textwidth}
	\begin{itemize}
		\itemsep0em
		\item[] $p$
		\item[] $\bullet p \lor q$
	\end{itemize}
\end{varwidth}
\hspace{4em}
\begin{varwidth}[t]{.5\textwidth}
	\begin{itemize}
		\itemsep0em
		\item[] $q$
		\item[] $\bullet p \lor q$
	\end{itemize}
\end{varwidth}
\subsubsection{Specialization}
\begin{varwidth}[t]{.5\textwidth}
	\begin{itemize}
		\itemsep0em
		\item[] $p \land q$
		\item[] $\bullet p$
	\end{itemize}
\end{varwidth}
\hspace{4em}
\begin{varwidth}[t]{.5\textwidth}
	\begin{itemize}
		\itemsep0em
		\item[] $p \land q$
		\item[] $\bullet q$
	\end{itemize}
\end{varwidth}
\subsubsection{Elimination}
If you have two possibilities and rule one out, the other one \textit{must} be the case.\\
\begin{varwidth}[t]{.5\textwidth}
	\begin{itemize}
		\itemsep0em
		\item[] $p \lor q$
		\item[] $\sim q$
		\item[] $\bullet p$
	\end{itemize}
\end{varwidth}
\hspace{4em}
\begin{varwidth}[t]{.5\textwidth}
	\begin{itemize}
		\itemsep0em
		\item[] $p \lor q$
		\item[] $\sim p$
		\item[] $\bullet q$
	\end{itemize}
\end{varwidth}
\subsubsection{Transitivity}
Many arguments in mathematics contains chains of if-then statements, which logically flow from one statement to the next.
Simply put, if one statement implies a second and a second implies a third, we can conclude that the first statement implies the third.\\
\begin{varwidth}[t]{.5\textwidth}
	\begin{itemize}
		\itemsep0em
		\item[] $p \rightarrow q$
		\item[] $q \rightarrow r$
		\item[] $\bullet p \rightarrow r$
	\end{itemize}
\end{varwidth}

\subsubsection{Division into cases}
It often happens that you know one thing or another is true. If you can show that in either case a certain conclusion is true, then this conclusion must also be true.\\
\begin{varwidth}[t]{.5\textwidth}
	\begin{itemize}
		\itemsep0em
		\item[] $p \lor q$
		\item[] $p \rightarrow r$
		\item[] $q \rightarrow r$
		\item[] $\bullet r$
	\end{itemize}
\end{varwidth}
\subsection{Fallacies}
A fallay is an error in reasoning that results in an invalid argument.
There are a few common fallacies:
\begin{enumerate}
	\item Using ambiguous or false premises as if they were unambiguous or correct.
	\item Circular reasoning - assuming what is to be proved without deriving it from the premises.
	\item Jumping to a conclusion without adequate premises.
	\item Using the converse or the inverse of a statement to reach a conclusion.
\end{enumerate}
\subsection{Contradiction rule}
If you can show that the supposition that statement $p$ is false leads logically to a contradiction, then you can conclude that $p$ is true.\\ 
\\Recall that this is the entire point of the proof by contradiction.\\
\begin{varwidth}[t]{.5\textwidth}
	\begin{itemize}
		\itemsep0em
		\item[] $\sim p \rightarrow false$
		\item[] $\bullet p$
	\end{itemize}
\end{varwidth}
\newpage
\appendix
\section{Logical equivalences}
\begin{multicols}{3}
\begin{itemize}
	\item[] \textbf{Commutative laws}:
	\item[] \textbf{Associative laws}:
	\item[] \textbf{Distributive laws}:
	\item[] \textbf{Identity laws}:
	\item[] \textbf{Negation laws}:
	\item[] \textbf{Double negative law}:
	\item[] \textbf{Idempotent laws}:
	\item[] \textbf{Universal bound laws}:
	\item[] \textbf{De Morgan's laws}:
	\item[] \textbf{Absorption laws}:
	\item[] \textbf{Negation of true and false}:
\end{itemize}
\columnbreak
\begin{itemize}
	\item[] $p \land q \equiv q \land p$
	\item[] $(p \land q) \land r \equiv p \land (q \land r)$
	\item[] $p \land (q \lor r) \equiv (p \land q) \lor (p \land r)$
	\item[] $p \land true \equiv p$
	\item[] $p \lor \sim p \equiv true$
	\item[] $\sim (\sim p) \equiv p$
	\item[] $p \land p \equiv p$
	\item[] $p \lor true \equiv true$
	\item[] $\sim(p \land q) \equiv \sim p \lor \sim q$
	\item[] $p \lor (p \land q) \equiv p$
	\item[] $\sim true \equiv false$
\end{itemize}
\columnbreak
\begin{itemize}
	\item[] $p \lor q \equiv q \lor p$
	\item[] $(p \lor q) \lor r \equiv p \lor (q \lor r)$
	\item[] $p \lor (q \land r) \equiv (p \lor q) \land (p \lor r)$
	\item[] $p \lor false \equiv p$
	\item[] $p \land \sim p \equiv false$
	\item[]
	\item[] $p \lor p \equiv p$
	\item[] $p \land false \equiv false$
	\item[] $\sim(p \lor q) \equiv \sim p \land \sim q$
	\item[] $p \land (p \lor q) \equiv p$
	\item[] $\sim false \equiv true$
\end{itemize}
\end{multicols}
\section{Order of operations}
\begin{enumerate}
	\item $()$ parentheses
	\item $\sim$ not
	\item $\land, \lor$ and, or
	\item $\rightarrow, \leftrightarrow$ if-then/implies, if and only if
\end{enumerate}
\end{document}