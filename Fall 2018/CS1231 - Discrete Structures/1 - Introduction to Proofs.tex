\documentclass[a4paper]{article}

%% Language and font encodings
\usepackage[english]{babel}
\usepackage[utf8x]{inputenc}
\usepackage[T1]{fontenc}

%% Useful packages
\usepackage{amsmath}
\usepackage{amssymb}
\usepackage{amsfonts} 
\usepackage{graphicx}
\usepackage{listings}
\usepackage{color}
\usepackage[colorinlistoftodos]{todonotes}
\usepackage[colorlinks=true, allcolors=blue]{hyperref}

%% Defining commands
\newcommand{\R}{\mathbb{R}}
\newcommand{\Z}{\mathbb{Z}}
\newcommand{\Q}{\mathbb{Q}}
\newcommand\tab[1][1cm]{\hspace*{#1}}
\newcommand*{\perm}[2]{{}^{#1}\!P_{#2}}%
\newcommand*{\comb}[2]{{}^{#1}C_{#2}}%

%% Sets page size and margins
\usepackage[a4paper,top=3cm,bottom=2cm,left=3cm,right=3cm,marginparwidth=1.75cm]{geometry}

%% Setting up code blocks
\lstset{frame=tb,
	aboveskip=3mm,
	belowskip=3mm,
	showstringspaces=false,
	columns=flexible,
	basicstyle={\small\ttfamily},
	numbers=none,
	breaklines=true,
	breakatwhitespace=true,
	tabsize=3
}




\title{%
	CS1231 Part 1 - Introduction to Proofs \\
	\large Based on lectures by Terence Sim and Aaron Tan
	\\ Notes taken by Andrew Tan
	\\ AY18/19 Semester 1
	\\ }

\author{}
\date{\vspace{-5ex}}

\begin{document}
\maketitle

\begin{center}\begin{minipage}[c]{0.9\textwidth}\centering\footnotesize These notes are not endorsed by the lecturers, and I have modified them (often significantly) after lectures. They are nowhere near accurate representations of what was actually lectured, and in particular, all errors are almost surely mine.\end{minipage}\end{center}

\section{Proof Techniques}
A proof is a concise, polished argument explaining the validity of a statement.
\subsection{Proof by Construction}
A proof by construction demonstrates the existence of a mathematical object by creating the object.
\subsection{Proof by Counterexample}
Conversely, a proof by counterexample provides a mathematical object that disproves a given statement.
\subsection{Proof by Contraposition}
The contraposition of \begin{center}$if\ P\ then\ Q$\end{center} is \begin{center}$if\ \sim P\ then\ \sim Q$\end{center}
Both statements are logically equivalent, and hence you can prove one by proving the other.
\subsection{Proof by Contradiction}
Given a statement S, only one of the following is true:
\begin{center}$S\ is\ true$\\$ \sim S\ is\ true$\end{center}
Thus to prove a statement S by contradiction, you first assume $\sim$S is true. Then use facts and theorems to arrive at a contradiction, which then implies that $\sim$S is false, and hence S is true.

\section{Logical Statements}
\subsection{If-then}
Many statements in proofs have the following structure:\\ \begin{center}$if\ P\ then\ Q$ \end{center}
We can use direct proofs to prove statements of this form: We first assume P is true, then work forwards by combining P with other facts and theorems to conclude that Q is true.

\subsection{For-all}
The for-all statement, $\forall x\ P(x)$,
essentially states that P(x) is true for all x in a given set. To prove statements of this form, we prove that P(x) is true for a particular but arbitrary x. Since x is arbitrary (that is, x is no different to the other elements of the set it was taken from), we can conclude that P(x) is true for all x.

\appendix
\section{Definitions}
Definition 1.3.1 (Divisibility) - if $n$ and $d$ are integers and d $\neq$ 0, then $n$ is divisible by $d$ if, and only if, $n$ equals $d$ times some integer.
\begin{center} $d|n$ $\Longleftrightarrow$ $\exists k \in \Z, n=dk$ \end{center}
Definition 1.6.1 - An integer n is even if, and only if, n equals twice some integer.\\An integer n is odd if, and only if, n equals twice some integer plus 1.
\begin{center}$n$ is even $\Longleftrightarrow$ $\exists$ an integer $k$ such that $n=2k$\\
$n$ is odd $\Longleftrightarrow$ $\exists$ an integer $k$ such that $n=2k+1$\end{center}
\section{Theorems}
Theorem 4.3.1 (Epp) $\forall a,b \in Z_{+}$, if a$|$b then $a \le b$\\
Theorem 4.3.3 (Epp) - Transitivity of Divisibility
\begin{center} $\forall a,b,c \in Z$ if a$|$b and b$|$c, then a$|$c \end{center}
\section{Notation}
\subsection{Set Notation}
$\R$ - the set of real numbers\\
$\Z$ - the set of all integers\\
$\Q$ - the set of all rational numbers\\
\subsection{Logic Notation}
$\exists$ - there exists at least one\\
$\exists!|$ - there exists one and only one\\
$\forall$ - for all\\
$\in$ - is a member of\\
$\notin$ - is not a member of\\
$\ni$ - such that\\

\section{Properties of the Real Numbers}
\begin{enumerate}
  \item Closure: Integers are closed under addition and multiplication.\\ $\forall x,y \in \Z, x + y \in \Z,$ and $xy \in Z$\\ \\
  For all real numbers a,b, and c,\
  \item Commutativity: $a + b = b + a$ and $ab = ba$
  \item Distributivity: $a(b + c) = ab + ac$
  \item Trichotomy: exactly one of the following is true:\\
  $a < b$, $a > b$, or $a = b$
\end{enumerate}


\end{document}