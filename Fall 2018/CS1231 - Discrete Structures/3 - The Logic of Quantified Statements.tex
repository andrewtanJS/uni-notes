\documentclass[a4paper]{article}

%% Language and font encodings
\usepackage[english]{babel}
\usepackage[utf8x]{inputenc}
\usepackage[T1]{fontenc}

%% Useful packages
\usepackage{amsmath}
\usepackage{amssymb}
\usepackage{amsfonts} 
\usepackage{graphicx}
\usepackage{listings}
\usepackage{color}
\usepackage[colorinlistoftodos]{todonotes}
\usepackage[colorlinks=true, allcolors=blue]{hyperref}

%% Defining commands
\newcommand{\R}{\mathbb{R}}
\newcommand{\Z}{\mathbb{Z}}
\newcommand{\Q}{\mathbb{Q}}
\newcommand\tab[1][1cm]{\hspace*{#1}}
\newcommand*{\perm}[2]{{}^{#1}\!P_{#2}}%
\newcommand*{\comb}[2]{{}^{#1}C_{#2}}%

%% Sets page size and margins
\usepackage[a4paper,top=3cm,bottom=2cm,left=3cm,right=3cm,marginparwidth=1.75cm]{geometry}

%% Setting up code blocks
\lstset{frame=tb,
	aboveskip=3mm,
	belowskip=3mm,
	showstringspaces=false,
	columns=flexible,
	basicstyle={\small\ttfamily},
	numbers=none,
	breaklines=true,
	breakatwhitespace=true,
	tabsize=3
}




\title{%
	CS1231 Part 3 - The Logic of Quantified Statements  \\
	\large Based on lectures by Terence Sim and Aaron Tan
	\\ Notes taken by Andrew Tan
	\\ AY18/19 Semester 1
	\\ }

\author{}
\date{\vspace{-5ex}}

\begin{document}
\maketitle

\begin{center}\begin{minipage}[c]{0.9\textwidth}\centering\footnotesize These notes are not endorsed by the lecturers, and I have modified them (often significantly) after lectures. They are nowhere near accurate representations of what was actually lectured, and in particular, all errors are almost surely mine.\end{minipage}\end{center}

\section{Predicates and quantified statements}
A \textbf{predicate} is a sentence that contains a finite number of values and becomes a statement when specific values are substituted for the variables.\\
The \textbf{domain} of a predicate variable is the set of all values that may be substituted in place of the variable.\\
The \textbf{truth set} of a predicate $P(x)$, where $x$ has a domain $D$, is the set of all elements of $D$ that make $P(x)$ true when they are substituted for $x$. The truth set of $P(x)$ is denoted
\begin{center}
	$\{x \in D$  $|$  $P(x)\}$
\end{center}
\subsection{The universal quantifier}
The symbol $\forall$, denoted "for all", is called the \textbf{universal quantifier}.\\ \\
Let $Q(x)$ be a predicate and $D$ the domain of $x$. A \textbf{universal statement} is a statement of the form 
\begin{center}
	$\forall x \in D, Q(x)$
\end{center}
It is defined to be true iff $Q(x)$ is true for every $x$ in $D$, and it is defined to be false iff $Q(x)$ is false for at least one $x$ in $D$.\\
A value for $x$ for which $Q(x)$ is false is called a \textbf{counterexample}.\\ \\
The \textbf{method of exhaustion} proves that a universal statement is true by exhausting all cases or proving that the statement is true for each element in the domain.
\subsection{The existential quantifier}
The symbol $\exists$, denoted "there exists", is called the \textbf{existential quantifier}.\\ \\
Let $Q(x)$ be a predicate and $D$ the domain of $x$. An \textbf{existential statement} is a statement of the form 
\begin{center}
	$\exists x \in D$ such that $Q(x)$
\end{center}
It is defined to be true iff $Q(x)$ is true for at least one $x$ in $D$, and it is false iff $Q(x)$ is false for all $x$ in $D$.\\ \\ 
Furthermore, the symbol $\exists!$ is used to denote "there exists a unique" or "there is one and only one". 
\subsection{Universal conditional statements}
The \textbf{universal conditional statement} comes in the form of:
\begin{center}
	$\forall x$, if $P(x)$ then $Q(x)$
\end{center}
\subsubsection{Equivalent forms of universal and existential statements}
Given a statement $\forall x \in U,$ if $P(x)$ then $Q(x)$, we can narrow the domain $U$ to be the domain $D$ consisting of all values of the variable $x$ that make $P(x)$ true:
\begin{center}
	$\forall x \in U,$ if $P(x)$ then $Q(x) \equiv \forall x \in D$, $Q(x)$
\end{center} 
This works similarly for existential statements.
\subsection{Implicit quantification}
Let $P(x)$ and $Q(x)$ be predicates and supposed the common domain of $x$ is $D$.\\
$\bullet$ The notation $P(x) \implies Q(x)$ means that every element in the truth set of $P(x)$ is in the truth set of $Q(x)$, or, equivalently, $\forall x, P(x) \rightarrow Q(x)$.\\
$\bullet$ The notaion $P(x) \iff Q(x)$ means that the truth sets of $P(x)$ and $Q(x)$ are identical, or, equivalently, $\forall x, P(x) \leftrightarrow Q(x)$

\subsection{Negations of quantified statements}
Negation of a universal statement:
\begin{center}
	$\sim(\forall x$ in $D$, $P(x))$ $\equiv$ $\exists x$ in $D$ such that $\sim P(x)$
\end{center}
Negation of an existential statement:
\begin{center}
	$\sim(\exists x$ in $D$ such that $P(x))$ $\equiv$ $\forall x$ in $D$, $\sim P(x)$
\end{center}
Negation of universal conditional statements:
\begin{center}
	$\sim (\forall x$, $P(x) \rightarrow Q(x))$ $\equiv$ $\exists x$ such that $P(x) \land \sim Q(x)$
\end{center}

\subsection{Variants of universal conditional statements}
Consider a statement of the form: $\forall x \in D$, if $P(x)$ then $Q(x)$
\begin{enumerate}
	\item Its contrapositive is: $\forall x \in D$, if $\sim Q(x)$ then $\sim P(x)$
	\item Its converse is: $\forall x \in D$, if $Q(x)$ then $P(x)$
	\item Its inverse is: $\forall x \in D$, if $\sim P(x)$ then $\sim Q(x)$
\end{enumerate}
\section{Statements with multiple quantifiers}
A statement may have multiple quantifiers, and the meaning of the statement depends on the quantifiers used, and their order within the statement.

\subsection{Negation of multiply-quantified statements}
We can use the equivalencies for the negation of statements with only one quantifier to deduce the negations of multiply quantified statements. For example:\\
As $\sim(\forall x$ in $D$, $P(x)) \equiv \exists x$ in $D$ such that $\sim P(x)$,\\ and $\sim (\exists x$ in $D$ such that $P(x))$ $\equiv$ $\forall x$ in $D, \sim P(x)$,
\begin{center}
	$\sim (\forall x$ in $D$, $\exists y$ in $E$ such that $P(x,y))$ $\equiv$ $\exists x$ in $D$ such that $\forall y$ in  $E$, $\sim P(x,y)$
\end{center}
As a general rule, to negate a quantified statement, negate all statement variables, and switch all $\forall$'s and $\exists$'s while maintaining their order.


\section{Arguments with quantified statements}
The rule of \textbf{universal instantiation} states that if some property is true of \textit{everything} in the set, then it is true of \textit{any particular thing} in the set.
This rule is the fundamental tool for deductive reasoning.\\ \\
With this, we can obtain the valid form of argument, \textbf{universal modus ponens}:
\begin{center}
	$\forall x$, if $P(x)$ then $Q(x)$\\
	$P(a)$ for a particular $a$.\\
	$\bullet$ $Q(a)$.
\end{center}
Likewise, we can obtain the valid form of argument, \textbf{universal modus tollens}:
\begin{center}
	$\forall x$, if $P(x)$ then $Q(x)$\\
	$\sim Q(a))$ for a particular $a$.\\
	$\bullet$ $\sim P(a)$
\end{center}
Furthermore, we can create additional forms of arguments involving universally quantified statements simply through \textbf{universal transitivity}:
\begin{center}
	$\forall x$, $P(x) \rightarrow Q(x)$\\
	$\forall x$, $Q(x) \rightarrow R(x)$\\
	$\bullet$ 	$\forall x$, $P(x) \rightarrow R(x)$\\
\end{center}
\end{document}